\section{Discussion and Future Timetable}(Matthew)

In this paper we have described the new powerful backend for the Molonglo radio telescope dubbed the "UTMOST". Science operations have commenced, but the commissioning of the telescope to date has been restricted to just half the bandwidth (16 MHz) and 88 of the 352 modules. 

New more energy efficient and powerful GPUs are almost always on the horizon, and to date we have avoided purchasing the final stage of the backend while other issues are being resolved.

The two remaining issues with the facility are self-generated digital noise from the RX boxes that adds to the system noise and the servicing of the ring antennae. 

The RX boxes have been shown to emit deleterious emissions that make fringes on short baselines corrupted and raise the system temperature by sometimes a 1000K or more. A great deal of progress has been made on this in recent weeks and a large-scale refurbishment is being planned for Q4 2014.

The 7744 ring antennae are over 30 years old, and require regular servicing. In Q1 2015 all of the modules have been serviced and the last computers and GPUs for the system installed, while the remaining work on the receiver boxes is expected to continue through Q2-Q3 2015. We anticipate that the system's sensitivity will gradually approach the theoretical peak by Q4 2015.