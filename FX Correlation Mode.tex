\subsection{Corner Turned Modes}
\subsubsection{Tied Array Beam Mode}

Tied array beam mode presently (April 2015) can be operated on half of the antennae (176 modules) over half final the bandwidth (15 MHz). A tied array beam can be formed at an arbitrary sky position and an object tracked, or placed at an arbitrary position and objects allowed to transit through it. A tied array beam formed on the Vela pulsar, with the data written to disk after square law detection and integration up to 1 ms time resolution. We show in Figure X a selection of individual pulses detected after dedispersion of these data. Pulsars can also be folded according to their current ephemerides while operating the tied array beam (cf. section 6.1) 

\subsubsection{FX Correlation Mode}

In Figure X+3 we show the first map made with the system of the field surrounding the quasar XXXX-YYYY using 88 modules. The noise in the background image is XX mJy and the dynamic range better than XX dB.

\subsubsection{Fan Beam Mode}

The raw voltage dump mode was used to record voltages for 2 minutes on the field containing the Vela pulsar as it drifted through the primary beam. A series of fan beams were created at differing delays and passed through standard single pulse dedispersion codes. Figure Z shows the pulsar drifting in and out of a fan beam and being detected at the correct dispersion measure.

