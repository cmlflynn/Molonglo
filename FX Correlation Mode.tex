\subsection{Corner Turned Modes}
\subsubsection{Tied Array Beam Mode}

Tied array beam mode was upgraded to operation on all 352 antennae and 40 frequency channels with the installation of the BF servers (May 2015). A tied array beam (TB) can be formed at an arbitrary sky position and an object tracked, or placed at an arbitrary position and objects allowed to transit through it. We show in Figure X a selection of individual pulses detected after dedispersion of data taken in this mode on Vela. Pulsars can also be folded according to their current ephemerides while operating the tied array beam (cf. section 6.1) 

\subsubsection{FX Correlation Mode}

In Figure X+3 we show the first map made with the UTMOST, of a 1 square degree field centered on the bright Souther radio source Fornax A, taken in November 2014. The noise in the background image is XX mJy and the dynamic range better than XX dB. Only 88 modules, or 1/4 of the telescope collecting area were used, as this was the limit at the time of data acquisition (full data rate acquisition was achieved in May 2015). The image noise 5 mJy and is dominated by remaining artifacts rather than thermal noise. Significant improvements are expected to our mapping capabilities from Q3-2015, as commissioning of remaining modules proceeds. 

\subsubsection{Fan Beam Mode}

In April 2015 we achieved full tiling of the primary beam with a set of 352 narrow "fan beams". The fanbeams are aligned East-West, covering 4 degrees, and have the same resolution in the N-S direction as the primary beam (2 degrees). Their East-West resolution is this 4/352 $\approx 0.7 arcmin$. 

Figure Z shows operational tests of this mode; the telescope was pointed at the meridian and at the declination of Vela, and the pulsar allowed to transit through the 352 fan beams. Burst detection software (Heimdall) running on all 352 fanbeams isolated events on times scales from 1 to approx 32 ms, in time series data dedispersed for a set of 120 DM values in the range 0 to 1500 pc/cm$^-3$. Detected bursts are shown in grey (SNR<10), green (SNR>10) and blue (for those identified as Vela, on the basis of its known DM). A burst of handset generated RFI is clearly seen as the vertical strip of grey and green marked events. Individual pulses from Vela had an SNR of $\approx 40$ at the time of taking the data (May 2015), and virtually all of its pulses are detected as it crosses through the fanbeams. RFI can be readily recognised as bursts occuring simultaneously in 3 or more fanbeams -- this turns out to be a very efficient means of removing RFI as par of our Fast Radio Burst search program.    
