\section{Introduction} 
The Molonglo Observatory Synthesis Telescope (MOST) is the 18,000 m$^2$ East-West arm of the original Mills Cross telescope situated in a valley just 40 km east of Australia's capital city, Canberra. The original Mills Cross played a pivotal role in the early years of radio pulsar astronomy. It discovered the Vela pulsar \scite{LARGE_1968} and performed the second Molonglo survey which discovered 155 objects, more than doubling the known population at the time \cite{Manchester_1978}. As a synthesis array, the MOST created many important catalogues, including the SUMSS \cite{Bock_1999,Mauch_2003} and supernova remnant \cite{Whiteoak_1996} ones. It was also useful for rapid follow-up observations such as the detection of the prompt radio emission from supernova 1987A \cite{Turtle_1987} and the discovery of radio emission from the micro-quasar GRO 1655-40 \cite{Tingay_1995}. Patient monitoring led to the re-detection of SN 1987A as a radio supernova remnant \cite{Staveley_Smith_1992}.

The original signal processor for the instrument \cite{Robertson_1991} was designed in the 1980s when the exorbitant cost of digital signal processing often placed strong constraints on the capability of an interferometer. The MOST processed a single 3 MHz band from each of $N=88$ "bays" at a centre frequency of 843 MHz by producing 64 fan-beams created by multiplying each arm against the other in a multiplication interferometer. 

In 2004 a series of upgrades was proposed for the instrument using a series of custom boards using Field Programmable Gate Array (FPGA) technology, optical fibres and a new digital correlator\cite{Adams_2004}. The Square Kilometre Array Molonglo Pathfinder (SKAMP) upgrades to the MOST telescope were first to replace the current continuum correlator with an 8-bit digital one (SKAMP-1), then expand the system to 30 MHz of bandwidth (SKAMP-2) and increase the field of view by a factor of four by breaking the existing bays into four "modules".

SKAMP-2 required the production of 88 new digital receivers, over 200 km of optical fibre, a polyphase filter-bank, a custom mesh, correlator and a final fine stage of channelisation before being written out to a computer for offline processing.

In early 2013 we commenced work on an alternative correlator solution using commodity-off-the-shelf components but retaining the original signal path up until the first polyphase filterbank. 

ultimately going to deliver spectral-line imaging across 30 MHz of bandwidth on 351 independent baselines\cite{Adams_2004}. The SKAMP-2 design was 


had 88 equivalent "bays" that provided 3 MHz of right-circularly polarised Rather than form all $N(N-1)/2$ baselines, where $N$ is the number of inputs ($N=88$ for the MOST), it created 64 fan beams using a multiplication interferometer between the two halves. Such interferometers have their sensitivity reduced by a factor $2^{-1/2}$ 

only capable of processing 3MHz of bandwidth from a single polarisation and used a multiplication interferometer to create up to 64 fan beams after 2-bit digitisation from 88 independent sections (bays) of the telescope. Despite these limitations, for many years the telescope occupied a niche in the Southern hemisphere, producing a series of important catalogues and  

