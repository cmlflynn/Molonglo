\section{Introduction} 
The Molonglo Observatory Synthesis Telescope (MOST) is the 18,000 m$^2$ East-West-aligned arm of the original Mills Cross telescope, situated in a valley 40 km east of Australia's capital city, Canberra. The Mills Cross played a pivotal role in the early years of radio pulsar astronomy. It discovered the Vela pulsar \citep{Large_1968} and performed the Second Molonglo Pulsar Survey, discovering 155 of the objects, and more than doubling the known population at the time \citep{Manchester_1978}. As a synthesis array, the MOST created many important catalogues, including the Sydney University Molonglo Sky Survey (SUMSS) \citep{Bock_1999,Mauch_2003} and the MOST Supernova Remnant Catalogue (MSC) \citep{Whiteoak_1996}. It is particularly useful for rapid follow-up observations of transient sources, such as the detection of the prompt radio emission from supernova SN1987A \citep{Turtle_1987} and the discovery of radio emission from the micro-quasar GRO 1655-40 \citep{Tingay_1995}. Patient monitoring led to the re-detection of SN 1987A as a radio supernova remnant \citep{Staveley_Smith_1992}.

A signal processor \citep{Robertson_1991} for the instrument was designed in the 1980s, at a time when the exorbitant cost of digital signal processing placed strong constraints on the capabilities of any interferometer. The MOST system was capable of processing a single 3 MHz band of data streamed from each of 88 ``bays'' (44 in each arm) at a centre frequency of 843 MHz by placing 64 fan-beams on the sky, created by multiplying each arm against the other in a "multiplication interferometer" design. The two-bit samplers and the fact that there were fewer fan-beams (64) than inputs (88), meant that modern methodologies such as "self-cal" could not be employed, so that the typical dynamic range in imaging mode was $\sim$20 dB \citep{Robertson_1991}. Away from bright sources, MOST's sensitivity was confusion rather than sensitivity limited.

\subsection{The SKAMP Upgrades}
In 2004, three upgrades were proposed for the instrument in the "Square Kilometre Array Molonglo Pathfinder" (SKAMP) project \cite{Adams_2004}, which would deploy Field Programmable Gate Array (FPGA) technology, involve long optical fibre feeds and the construction of two digital correlators \citep{Adams_2004}. 

The first of these upgrades, SKAMP-1, was to replace a continuum correlator with an 8-bit digital device. SKAMP-2 would increase the system bandwidth from 3 to 30 MHz and increase the field of view by a factor of four (by processing all 352 outputs of the telescope rather that combining them 4 at a time into 88 telescope sub-sections (termed "bays").

SKAMP-2 required the production of 88 new digital receivers (RX boxes), capable of handling input form 4 antennae each and the laying of over 200 km of optical fibre. The 88 RX boxes were to feed 22 polyphase filterbanks utilising a custom mesh, a new correlator and a final fine channelisation stage. To retain modest output data rates, redundant baselines were to be summed before being written to disk for offline processing \citep{Adams_2004}. Unfortunately, sourcing and programming an appropriate correlator proved difficult, and the limited opportunities for monitoring such a system would have made debugging quite difficult, leading to project delays.

Other issues also arose. In Australia, the telecommunications industry acquired sections of the 800-900 MHz spectrum for mobile phone use. Although the instrument site is a flat valley surrounded by low hills with a low population density, mobile phone calls became are a major source of RFI, and can dominate the system noise on times scales of 20 milliseconds (for handset registrations) to seconds to tens of minutes (phone calls) in several $\sim$5 MHz bands within the intended 31.25 MHz band of SKAMP-2. SKAMP-2's original correlator design proposed to excise this interference via a reference antennae, which would integrate for 10 or more seconds and flag for the presence of FRI. 

By mid-2013, although sub-sections or prototypes of the signal processing chain existed, end-to-end tests of the system were yet to be performed, and the custom nature of many of the boards meant that tools to tap off data for debugging were limited. New radio projects like the ASKAP telescope were also on the horizon, and the instrument's future was in doubt.

    
    