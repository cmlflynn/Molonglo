\section{Introduction} 
The Molonglo Observatory Synthesis Telescope (MOST) is the 18,000 m$^2$ East-West arm of the original Mills Cross telescope situated in a valley just 40 km east of Australia's capital city, Canberra. The original Mills Cross played a pivotal role in the early years of radio pulsar astronomy. It discovered the Vela pulsar \citep{LARGE_1968} and performed the second Molonglo survey which discovered 155 objects, more than doubling the known population at the time \citep{Manchester_1978}. As a synthesis array, the MOST created many important catalogues, including the SUMSS \citep{Bock_1999,Mauch_2003} and supernova remnant \citep{Whiteoak_1996} ones. It was also useful for rapid follow-up observations such as the detection of the prompt radio emission from supernova 1987A \citep{Turtle_1987} and the discovery of radio emission from the micro-quasar GRO 1655-40 \citep{Tingay_1995}. Patient monitoring led to the re-detection of SN 1987A as a radio supernova remnant \citep{Staveley_Smith_1992}.

The original signal processor \citep{Robertson_1991} for the instrument was designed in the 1980s when the exorbitant cost of digital signal processing often placed strong constraints on the capability of any interferometer. The MOST processed a single 3 MHz band from each of $N=88$ "bays" at a centre frequency of 843 MHz by producing 64 fan-beams created by multiplying each arm against the other in a multiplication interferometer. The two-bit samplers and fewer fan-beams (64) than inputs (88) meant that modern methodologies such as self-cal could not be employed, and the typical dynamic range in images was $\sim$20 dB (REFERENCE). Away from bright sources, the sensitivity was limited by confusion.

In 2004 a series of upgrades was proposed for the instrument using a series of custom boards using Field Programmable Gate Array (FPGA) technology, optical fibres and a new digital correlator\citep{Adams_2004}. The Square Kilometre Array Molonglo Pathfinder (SKAMP) upgrades to the MOST telescope were first to replace the current continuum correlator with an 8-bit digital one (SKAMP-1), then expand the system to 30 MHz of bandwidth (SKAMP-2) whilst increasing the field of view by a factor of four by breaking the existing bays into four ``modules''.

SKAMP-2 required the production of 88 new digital receivers and the laying of over 200 km of optical fibre. These were to feed a 22 polyphase filterbanks that used a custom mesh, new correlator and a final fine stage of channelisation. Redundant baselines were to be summed before being written out to a computer for offline processing \citep{Adams_2004}. Sourcing and programming an appropriate correlator proved difficult, and the limited opportunities for monitoring made debugging the system components difficult, leading to project delays.

Other issues also arose. In Australia the 800-900 MHz part of the spectrum is now dedicated to mobile phone telecommunications. Although the instrument is shielded by a valley, mobile phone calls in the vicinity do occur relatively often, and can often dominate the system noise for anything from 20 milliseconds to tens of minutes in $\sim$4 MHz bands within the intended 31.25 MHz band of SKAMP-2. SKAMP-2's original correlator design was hoping to excise these signals by using a reference antennae, but only after integrating for 10 or more seconds. The system design meant that higher time resolution for interference excision would further complicate the digital signal path. 

Although sub-sections or prototypes of the signal processing chain existed, it was impossible to perform end-to-end tests of the system, and the custom nature of many of the boards meant that tools to tap off data were limited.  New radio projects like the ASKAP telescope were also on the horizon, and the instrument's future was in doubt.

Well after the SKAMP-2 project commenced a new class of radio transient emerged, that of the Fast Radio Bursts or FRBs. \cite{Lorimer_2007} reported the first FRB, a dispersed 30-Jy pulse of potentially extra-galactic origin just 5 milliseconds in duration after examining archival data from the Parkes 64\,m radio telescope. Six years later \citep{Thornton_2013} reported four more FRBs, with dispersion measures ranging up to 1100 pc cm$^{-3}$. But finding FRBs is extremely difficult. Only 8 are now in the literature, and all have been detected by single dishes. This means their spatial localisation is limited to a few tens of arcminutes, insufficient to unambiguously link to any host galaxy.

The FRBs are also not without their controversy. \cite{Burke_Spolaor_2011} detected swept signals at the Parkes telescope with similar dispersion to the original ``Lorimer burst'', and it will not be until they are detected by an interferometer that their celestial nature will be confirmed. Intriugingly, in the late 1980s a transient event detector found a series of millisecond-timescale impulsive radio bursts with the MOST which they concluded were occurring many 1000 km from the site up to 18,000 times per sky per day\cite{Lovell_2008}. Their lack of frequency resolution meant they couldn't comment about their dispersion but their event rate was similar to that implied by \citep{Thornton_2013} (10,000 sky$^{-1}$ day$^{-1}$) for the FRBs. The major scientific driver for the MOST is to determine if the AMYetal events were FRBs, and if so to determine their dispersion measure location.

A new trend in radio astronomy has emerged over the past 10-15 years, that of the ``software correlator''. Software correlators are less energy efficient than dedicated chips or Field Programmable Gate Arrays (FPGAs) but are extremely quick to deploy, and use commodity "off the shelf" components that are cheap to source and replace. Examples of software correlators for pulsar observations are the pulsar processors CPSR2 \citep{Bailes_2009} and GUPPI\citep{DuPlain_2008}. It is possible to publish software instruments such as the pulsar processor \sc{dspsr} \citep{van_Straten_2011} and make the source code available for quick installation by other users. \citep{Deller_2007} published {\bf DiFX} a software correlator for Very Long Baseline Interferometry and \sc{xGPU} \citep{Clark_2012} is a GPU-based software correlator that is being used in many new "large-N" interferometers such as the MWA \citep{Tingay_2013}. 

The MOST's wide field of view (about 8 square degrees), large collecting area (18,000 m$^2$) and compact configuration made it an attractive instrument with which to search for FRBs. The radiometer equation provides an estimate of the system noise $N$ in an integration of length $t$ 
\begin{equation}
N={{T_{\rm sys}}\over{G\sqrt{BN_{\rm p}t}}}
\end{equation}
\noindent $B$ is the bandwidth, $N_{\rm p}$ is the number of polarisations (for the MOST $N_{\rm p}=1$)and $G$ is the gain of the antenna in K Jy$^{-1}$. For the MOST $G\sim 3.5$ K Jy$^{-1}$. The effective bandwidth is considerably less than the 31.25 MHz that is processed due to optimisation for the previous 3 MHz band. An FRB of duration $t=5$\, ms if we assume $T_{\rm sys}$=60K and $B=15 $ MHz the noise is $\sim$60 mJy. This is enough to detect the brightest of the bursts present in Thornton et al. (2013) about about the 8 sigma level but the UTMOST has about 15 times the field of view of the Parkes multibeam receiver.
A software backend was proposed in early 2013 after it was realised we could tap off the data fairly early on in the digital signal processing chain, keeping the first two stages of the SKAMP-2 system but doing the rest of the processing on commodity CPUs and GPUs. UTMOST is designed to provide the ultimate in processing flexibility and provide commensal modes to enable simultaneous mapping, pulsar coherent dedispersion, burst-searching and baseband dump modes whilst employing novel intereference mitigation strategies.

This paper describes the scientific capability and design of the ``UTMOST'', the Swinburne University of Technology CPU/GPU backend for the MOST. In section 2 we provide a block diagram and a give brief description of all of the system components. Examples of the new data monitoring capabilities in section 3. The site is subject to interference from domestic mobile phones and communications towers, and a novel technique to detect and eliminate this is described in section 4. Examples of the major observing modes are provided in section 5, and the major initial science programs in section 6. Finally, in section 7 we describe the remaining issues with the instrument and the hardware timetable for completion.



