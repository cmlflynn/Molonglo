\section{Introduction} 
The Molonglo Observatory Synthesis Telescope (MOST) is the 18,000 m$^2$ East-West arm of the original Mills Cross telescope situated in a valley just 40 km east of Australia's capital city, Canberra. The original Mills Cross played a pivotal role in the early years of radio pulsar astronomy. It discovered the Vela pulsar \citep{LARGE_1968} and performed the second Molonglo survey which discovered 155 objects, more than doubling the known population at the time \citep{Manchester_1978}. As a synthesis array, the MOST created many important catalogues, including the SUMSS \citep{Bock_1999,Mauch_2003} and supernova remnant \citep{Whiteoak_1996} ones. It was also useful for rapid follow-up observations such as the detection of the prompt radio emission from supernova 1987A \citep{Turtle_1987} and the discovery of radio emission from the micro-quasar GRO 1655-40 \citep{Tingay_1995}. Patient monitoring led to the re-detection of SN 1987A as a radio supernova remnant \citep{Staveley_Smith_1992}.

The original signal processor \citep{Robertson_1991} for the instrument was designed in the 1980s when the exorbitant cost of digital signal processing often placed strong constraints on the capability of any interferometer. The MOST processed a single 3 MHz band from each of 88 "bays" at a centre frequency of 843 MHz by producing 64 fan-beams created by multiplying each arm against the other in a multiplication interferometer. The two-bit samplers and fewer fan-beams (64) than inputs (88) meant that modern methodologies such as self-cal could not be employed, and the typical dynamic range in images was $\sim$20 dB (REFERENCE). Away from bright sources, the sensitivity was limited by confusion.

\subsection{The SKAMP Upgrades}
In 2004 three upgrades (SKAMP 1-3) were proposed for the instrument that were to use a series of custom boards using Field Programmable Gate Array (FPGA) technology, and involve optical fibres (SKAMP 2\&3) and two new digital correlators \citep{Adams_2004}. The first Square Kilometre Array Molonglo Pathfinder (SKAMP) upgrade was to simply replace the current continuum correlator with an 8-bit digital one. SKAMP-2 was to expand the system to 30 MHz of bandwidth whilst increasing the field of view by a factor of four by breaking the existing bays into four ``modules''.

SKAMP-2 required the production of 88 new digital receivers and the laying of over 200 km of optical fibre. These were to feed 22 polyphase filterbanks that used a custom mesh, new correlator and a final stage of fine channelisation. To keep the data output rate modest, redundant baselines were to be summed before being written out to a computer for offline processing \citep{Adams_2004}. Sourcing and programming an appropriate correlator proved difficult, and the limited opportunities for monitoring made debugging the system components difficult, leading to project delays.

Other issues also arose. In Australia the 800-900 MHz part of the spectrum is now dedicated to mobile phone telecommunications. Although the instrument is shielded by a valley, mobile phone calls in the vicinity do occur relatively often, and can often dominate the system noise for anything from 20 milliseconds to tens of minutes in $\sim$4 MHz bands within the intended 31.25 MHz band of SKAMP-2. SKAMP-2's original correlator design was hoping to excise these signals by using a reference antennae, but only after integrating for 10 or more seconds. The system design meant that higher time resolution for interference excision would further complicate the digital signal path. 

Although sub-sections or prototypes of the signal processing chain existed, it was impossible to perform end-to-end tests of the system, and the custom nature of many of the boards meant that tools to tap off data for debugging were limited.  New radio projects like the ASKAP telescope were also on the horizon, and the instrument's future was in doubt.
