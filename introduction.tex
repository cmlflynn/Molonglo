\section{Introduction} 
The Molonglo Observatory Synthesis Telescope (MOST) is the 18,000 m$^2$ East-West arm of the original Mills Cross telescope situated in a valley just 40 km east of Australia's capital city, Canberra. The original Mills Cross played a pivotal role in the early years of radio pulsar astronomy. It discovered the Vela pulsar \cite{LARGE_1968} and performed the second Molonglo survey which discovered 155 objects, more than doubling the known population at the time \cite{Manchester_1978}. As a synthesis array, the MOST created many important catalogues, including the SUMSS \cite{Bock_1999,Mauch_2003} and supernova remnant \cite{Whiteoak_1996} ones. It was also useful for rapid follow-up observations such as the detection of the prompt radio emission from supernova 1987A \cite{Turtle_1987} and the discovery of radio emission from the micro-quasar GRO 1655-40 \cite{Tingay_1995}. Patient monitoring led to the re-detection of SN 1987A as a radio supernova remnant \cite{Staveley_Smith_1992}.

The original signal processor for the instrument \cite{Robertson_1991} was only capable of processing 3MHz of bandwidth from a single polarisation and used a multiplication interferometer to create up to 64 fan beams after 2-bit digitisation from 88 independent sections (bays) of the telescope. Despite these limitations, for many years the telescope occupied a niche in the Southern hemisphere, producing a series of important catalogues and  

