
The MOST consists of two parabolic cylinders 11.7\,m wide and 778\,m long separated by 15\,m. This parabolic cylinders enable a linear scaling of cost with collecting area but make steering difficult.

\subsection{The ``Modules''}
The MOST's solution to the east-west steering problem is an ingenius system of ring antennae that are sensitive to right-hand (IEEE) circularly polarised light. The ring antennae are differentially rotated to steer the phased-array beam and coherently add incident radiation in groups of 22 antennae spanning approximately 4.7\,m. These groups of 22 antennae are referred to as ``modules", and each arm possesses 176 of them. 

In the north-south direction the telescope can tilt and observe all sources south of $\delta<17^\circ$. The gain of an independent module is approximately 0.01 K Jy$^{-1}$ and the system temperature in the best modules is estimated to be of order 60K, although currently varies a great deal as sources of self-generated radio frequency interference are still being identified and removed. The LNAs are thought to only contribute 20-30K to the total system temperature. The gain of the system varies in a complicated manner due to the meridian distance gain curve that arises because of reflections along the cylinder and attenuate the signal so that the effective gain is $<$0.01 K Jy$^{-1}$\cite{Hunstead_1996}.
Four adjacent modules form a ``bay'' and have their independent signals are passed via shielded cables to the digital receiver or (RX) box. Up until this point the UTMOST is entirely equivalent to the old MOST. The MOST used to coherently add four modules together to form a ``bay'', which was filtered and brought back to the central signal processor in the control room. SKAMP-2's design sought to retain all four inputs from a bay, thus quadrupling the field of view, and bring back the signal on optical fibres.
