
The MOST consists of two parabolic cylinders 11.7\,m wide and 776\,m long (CHECK). This parabolic east-west structure enables a linear scaling of cost with collecting area but makes steering in the east-west direction difficult.

\subsection{The ``Modules''}
The MOST's solution to the east-west steering problem is an ingenius systems of ring antennae that are sensitive to right-circularly polarised light. The ring antennae are rotated to steer the beam to coherently add incident radiation in groups of 22 antennae spanning approximately 4.7\,m. This is referred to as a ``module", and each arm possesses 176 of them. In the north-south direction the telescope can tilt and observe all sources south of $\delta<17^\circ$. The gain of an independent module is approximately 0.01 K Jy$^{-1}$ and the system temperature in the best modules is estimated to be of order 60K, although currently varies a great deal as sources of self-generated radio frequency interference are still being identified and removed. The LNAs are thought to only contribute 20-30K to the total system temperature.
The gain of the system varies in a complicated manner due to the meridian distance gain curve that arises because of reflections along the cylinder and attenuate the signal\cite{Hunstead_1996}.
Four adjacent modules form a ``bay'' and have their independent signals are passed via shielded cables to the digital receiver or (RX) box. Up until this point the UTMOST is entirely equivalent to the old MOST.
