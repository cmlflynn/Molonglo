\subsection{The ``Modules''}
The MOST's solution to the east-west steering problem is an ingenious system of ring antennae that is sensitive to right-hand (IEEE) circularly polarised light. The ring antennae are differentially rotated to steer the phased-array beam and coherently add incident radiation in groups of 22 antennae spanning approximately 4.7\,m of telescope length. These groups of 22 antennae are referred to as ``modules". Each arm has 176 modules, for 352 in all. In the north-south direction each arm of the telescope can be tilted permitting observations of all sources south of $\delta<+17^\circ$. 

The gain of each module is approximately 0.01 K Jy$^{-1}$ and the system temperature in the best modules is estimated to be of order 60 K (from observations of point source calibrators), although currently this varies a great deal, as sources of self-generated radio frequency interference are still being identified and removed. The LNAs are thought to only contribute 20-30 K (onthe basis of lab measurements) to the total system temperature, with the remainder coming from a variety of sources including mesh leakage, cable feeds, etc. The effective gain of each module varies in a complex way as it is steered east-west, a function termed the '`meridian distance gain curve'' --- it arises in parabolic antennae because of reflections along the cylinder that attenuate the signal so that the effective gain is $<$0.01 K Jy$^{-1}$ per module \cite{Hunstead_1996}. When added coherently all 352 modules of the antenna combine to have a gain of 3.5 K Jy$^{-1}$.

Four adjacent modules form a ``bay'' and have their independent signals passed via shielded cable to each bay's digital receiver, or "RX box". Prior to the RX boxes, the UTMOST signal feed is entirely equivalent to the old MOST. The old MOST coherently added four modules together to form a ``bay'', which was filtered and brought back to the central signal processor in the control room. SKAMP-2's design retains all four independent inputs from a bay, thus quadrupling the field of view, and brings back the digitized signal on optical fibres.
  
  