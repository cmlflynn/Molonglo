Well after the SKAMP-2 project commenced a new class of radio transient emerged, that of the Fast Radio Bursts or FRBs. \cite{Lorimer_2007} reported the first FRB, a dispersed 30-Jy pulse of potentially extra-galactic origin just 5 milliseconds in duration after examining archival data from the Parkes 64\,m radio telescope. Six years later \citep{Thornton_2013} reported four more FRBs, with dispersion measures ranging up to 1100 pc cm$^{-3}$. But finding FRBs is extremely difficult. Only 8 are now in the literature, and all have been detected by single dishes. This means their spatial localisation is limited to a few tens of arcminutes, insufficient to unambiguously link to any host galaxy.

The FRBs are also not without their controversial aspects. \cite{Burke_Spolaor_2011} detected swept signals at the Parkes telescope with similar dispersion to the original ``Lorimer burst'', and it will not be until they are detected by an interferometer that their alleged celestial nature will be confirmed. Intriugingly, in the late 1980s a transient event detector found a series of millisecond-timescale impulsive radio bursts with the MOST which they concluded were occurring many 1000 km from the site up to 18,000 times per sky per day\cite{AMY}. Their lack of frequency resolution meant they couldn't comment about their dispersion but their event rate was similar to that implied by \citep{Thornton_2013} (10,000 sky$^{-1}$ day$^{-1}$) for the FRBs. The major scientific driver for the UTMOST project is to determine if the AMYetal events were FRBs, and if so to determine their dispersion measure and sky location.

\subsection{Radio Astronomy in Software}
A new trend in radio astronomy has emerged over the past 10-15 years, that of the ``software correlator''. Software correlators are less energy efficient than dedicated chips or Field Programmable Gate Arrays (FPGAs) but are extremely quick to deploy, and use commodity ``off the shelf'' components that are cheap to source and replace. Examples of software correlators for pulsar observations are the pulsar processors CPSR2 \citep{Bailes_2009} and GUPPI\citep{DuPlain_2008}. It is possible to publish and distribute software instruments such as the pulsar processor \textsc{dspsr} \citep{van_Straten_2011} via online repositories making them available for quick installation by other users. \citep{Deller_2007} published \textsc{DiFX} a software correlator for Very Long Baseline Interferometry and \textsc{xGPU} \citep{Clark_2012} is a GPU-based software correlator that is being used in many new "large-N" interferometers such as the MWA \citep{Tingay_2013}. 
