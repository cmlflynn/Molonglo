\section{Radio Frequency Interference Excision}  

Detecting, excising and/or avoiding Radio Frequency Interference (RFI) has always been a major concern in radio astronomy. 

Very early on in the project we discovered that pulsar observations with the UTMOST backend were highly compromised by mobile phone transmissions to the point where the project was in serious danger of being abandoned, unless this highly prevalent source of RFI could be excised very well. In Figure 6a we show the pulsar profile formed from a 10 minute track of the Vela pulsar where 44 modules had their signals added incoherently after square law detection. The baseline of the pulsar profile is so corrupted that the data are useless for any meaningful science.

The MOST operates in the Australian mobile phone band and often detects phone calls in (typically) 5 MHz sections of its band with variable strength across the array. It is rare that more than two 5 MHz bands are detected at any one time, but about 10\% of the time one call can be seen, usually during the mornings and late afternoon, but calls can be detected at any time. Short pulses of 20 milliseconds in duration are almost always present however, and these are used by phones to register with the local tower.

  