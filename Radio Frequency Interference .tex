\section{Radio Frequency Interference Excision} (Andrew?)

Detecting, excising and/or avoiding Radio Frequency Interference (RFI) has always been a major problem in radio astronomy. The MOST operates in the Australian mobile phone band and often detects phone calls in 4 MHz sections of its band with variable strength across the array. It is extremely rare that more than two 4 MHz bands are detected at any one time, but about 10\% of the time one call can be seen, usually during the mornings and late afternoon, but calls can be detected at any time. Short pulses of 20 milliseconds in duration are almost always present however, and these are used by phones to register with the local tower.

The pressure on the radio spectrum has never been higher