\section{Data Monitoring }(Chris and Tim?)

The software nature of the UTMOST provides a high level of opportunity for data monitoring and inspection. The lattice FPGA on the RX board can send small snippets of data via Gb ethernet to the DAQ machines which can be plotted as bandpasses, time series and data histograms via a web interface. Similarly once a second the UDP capture system sends 512 voltages from each channel to a ring buffer where a series of plots are created, again available for scrutiny via a web interface.

At the other extreme, it is possible to record large (10-120s) bursts of the PFB raw voltages to the SSD drives of the DAQ computers for offline processing. The software backend can either operate in real time on the real voltage streams or in an offline mode on the recorded voltages. The power of this mode in debugging the instrument can not be understated. 
