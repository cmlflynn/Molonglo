\subsection{The Polyphase Filterbanks}
The fibres from 4 RX boxes (16 modules) are aggregated in a rear transmission module that takes the 8/10 encoded optical data and places it on Rocket IO to send it to the polyphase filterbank (PFB) boards. The PFBs have a CX4 port which is used to send 8K 10Gb ethernet packets to industry standard (copper) network interface cards. The PFB can be reprogrammed into one of three modes, pass-through, critically-sampled PFB and over-sampled PFB. The pass-through mode was pivotal in debugging the more complex modes but the default is the critically-sampled mode. The over-sampled mode is reserved for future use and may help eliminate the loss of sensitivity between frequency channels but comes at the cost of an increased data rate (by a factor 128/100). The filterbank modes use 8-tap polyphase filters and time tags the data using the station 1PPS as a reference. The data are 8-bit real, 8-bit imaginary. Currently the system produces 128 channels each 100/128 = 781.25 kHz wide. The user can select 40 of the 128 channels for passing to the data acquisition servers. In critically-sampled mode, the data rate is exactly 8 Gb/s (here 1 G=10$^9$).

Up until this point the hardware is identical to that originally planned for SKAMP-2. The only change is that the PFB firmware has been re-programmed to send the data out of the CX-4 port using the industry-standard 10 Gb ethernet protocol instead of placing it on SKAMP-2's custom mesh.

