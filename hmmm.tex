
At the other extreme, it is possible to record large (10 to 120 sec real time) bursts of the PFB raw voltages to the SSD drives of the DAQ computers for subsequent offline processing. The software backend can either operate in real time on the real voltage streams or in an offline mode on the recorded voltages. The usefulness of this mode in debugging the software instrument cannot be understated. A real-time correlator cannot usually play back data, and thus cannot perform ``difference'' experiments or confirm the validity of new features on proven input data sets. During the commissioning phases of the correlator, raw data captured in this way proved invaluable in debugging the system: it could be read back dozens of times if required.

When observing in FX correlation mode, a ``correlation triangle'' is plotted on our web interface (Figure 3), with which the user can display phase versus time for any baseline, and used to update the instrumental delays. Good automation has been achieved in measuring delays and phase offsets for each module (June 2015) from standard calibrators in a database around the sky, with the SEFDs and beam forming weights being computed live during a calibrator observation, to aid the user decide when appropriate accuracy has been acquired.  

Folded pulsar profiles are also shown live on our web interface in both independent modes and tied array beam modes. Figure 4 shows an observation of the millisecond pulsar PSR J0437-4715 observed in tied array beam mode.
  