\subsection{Radio Astronomy in Software}
A new trend in radio astronomy has emerged over the past 10-15 years, that of the ``software correlator''. Software correlators are less energy efficient than dedicated chips or Field Programmable Gate Arrays (FPGAs) but are extremely quick to deploy, and use commodity ``off the shelf'' components that are cheap to source and replace. Examples of software correlators (in reality coherent dedispersers) for pulsar observations are the pulsar processors CPSR2 \citep{Bailes_2009} and GUPPI \citep{DuPlain_2008}. It is possible to publish and distribute software instruments such as the pulsar processor \textsc{dspsr} \citep{van_Straten_2011} via online repositories making them available for quick installation by other users. Other examples of software baseband processing instruments are the VLBI correlator \textsc{DiFX} \citep{Deller_2007} 
 and \textsc{xGPU} \citep{Clark_2012} a GPU-based correlator for ``large-N'' interferometers such as the MWA \citep{Tingay_2013}. 

A software backend was proposed in early 2013 after it was realised that the MOST could be a powerful FRB discovery engine and that we could tap off the data fairly early on in the digital signal processing chain, keeping the first two stages of the SKAMP-2 system but performing the rest of the signal processing on commodity CPUs and GPUs. Software was designed to provide the ultimate in processing flexibility and provide commensal modes to enable simultaneous mapping, pulsar coherent dedispersion, burst-searching and baseband dump modes whilst employing novel interference mitigation strategies.