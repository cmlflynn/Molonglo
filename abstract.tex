We describe how the deployment of a modern cluster of server-class computers at the Molonglo Observatory Synthesis Telescope (MOST) has transformed the instrument into a powerful new facility (the UTMOST). The servers form part of a hybrid solution to the observatory's radio astronomy signal processing requirements. The emphasis on software and commodity-off-the-shelf hardware has enabled extremely rapid deployment and re-use of many powerful existing "software backends" already proven at other observatories. The UTMOST provides a dramatic increase in the scientific capabilities of the MOST. Its primary beam has four times the field of view and it uses an FX software correlator rather than a multiplication interferometer to improve the base sensitivity by a further factor of $2^{1/2}$. The ability to perform self-cal vastly improves the dynamic range of maps and it can produce an arbitrary number of spectral channels rather than the simple continuum mode of MOST. The total bandwidth has also improved by an order of magnitude. Inherent to the solution is the ability to run commensal modes. The UTMOST can simultaneously make maps, coherently dedisperse multiple pulsars in the primary beam, search coherent fan beams for dispersed single pulses out to dispersion measures of several 1000 pc cm$^{-3}$, excise radio frequency interference and offer a range of diagnostic information previously unobtainable. A playback capability facilitates rapid debugging of new modes. The telescope can be broken up into 352 independent sections that can be used to measure the beam-shape and performance of individual 4.7m elements of the array that help improve maintenance efficiency. A pulsar timing program has already commenced, and the telescope will soon begin searching for fast radio bursts (FRBs), rotating radio transients (RRATs), and performing synthesis imaging. All of the FRB events will be made immediately public and after cleaning, so will the bulk of the radio pulsars being monitored for glitches and other timing irregularities.
