We describe how the deployment of a modern cluster of server-class GPU (Graphics Processing Unit) computers at the Molonglo Observatory Synthesis Telescope (MOST) has transformed the instrument into a powerful new facility (the UTMOST). The servers form part of a hybrid solution to the observatory's signal processing requirements, replacing the corner-turner, correlator and fine filterbanks of the SKAMP-2 system. The emphasis on software and commodity-off-the-shelf hardware has enabled extremely rapid deployment and re-use of many powerful existing ``software backends'' already proven at other sites. The telescope now has four times the field of view and ten times the bandwidth of the original MOST. The new FX software correlator is vastly superior to the MOST's single-channel fan-beam multiplication interferometer, and includes novel interference rejection algorithms. The 8-bit digitisation and ability to perform self-calibration techniques greatly improve the dynamic range of maps and the UTMOST can produce an arbitrary number of spectral channels. Inherent to the solution is the ability to run commensal modes. The UTMOST is designed to simultaneously make maps, coherently dedisperse multiple pulsars in the primary beam, perform real-time searches of coherent fan beams for dispersed single pulses out to dispersion measures of several 1000 pc cm$^{-3}$, excise radio frequency interference and offer a range of diagnostic information previously unobtainable. A record and playback capability facilitates rapid debugging of new modes. A pulsar timing program has already commenced, and the telescope will soon begin searching for fast radio bursts (FRBs), rotating radio transients (RRATs), and performing synthesis imaging. All of the FRBs discovered will be made public immediately as will the bulk of the radio pulsar timing data. 
