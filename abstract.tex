We describe how the deployment of server-class GPU (Graphics Processing Unit) computers at the Molonglo Observatory Synthesis Telescope (MOST) has transformed the instrument into a powerful new facility (the UTMOST). The servers form part of a hybrid solution to the observatory's signal processing requirements, replacing the corner-turner, correlator and fine filterbanks of the planned SKAMP-2 MOST upgrade. The emphasis on software and commodity-off-the-shelf hardware has enabled extremely rapid deployment and re-use of many powerful existing ``software backends'' already proven at other sites. The telescope now has four times the field of view and ten times the bandwidth of the original MOST. The FX software correlator is vastly superior to the MOST's single-channel fan-beam multiplication interferometer, and includes novel interference rejection algorithms. Eight bit digitisation and the ability to perform self-calibration techniques greatly improve the dynamic range of maps. Inherent to the solution is the ability to run commensal modes. The UTMOST is designed to simultaneously make maps, coherently dedisperse multiple pulsars in the primary beam, perform real-time searches of coherent fan beams for dispersed single pulses out to dispersion measures of several 1000 pc cm$^{-3}$, excise radio frequency interference in real time and offer a range of diagnostic information on system performance, unobtainable with the former system. A record and playback capability facilitates rapid debugging of new modes. Pulsar timing program and Fast Radio Burst (FRBs) searches have commenced, as well as searches for rotating radio transients (RRATs) and a small program of 1 and 2-D synthesis imaging --- even if the system is in the last stages of commissioning. Full performance is expected in Q4, 2015.   
    
    