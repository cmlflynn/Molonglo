We describe the deployment of a modern cluster of server-class computers at the Molonglo Observatory Synthesis Telescope (MOST) that have transformed the instrument into a powerful new facility (UTMOST). The servers form part of a hybrid Field Programmable Gate Array (FPGA)/ Central Processing Unit (CPU)/ Graphics Processing Unit (GPU) solution to the radio astronomy signal processing problem. The emphasis on software has enabled extremely rapid deployment of the solution and re-use of many powerful existing "software backends" already proven at other observatories. UTMOST represents a dramatic increase in the scientific capabilities of the MOST. Its primary beam has four times the field of view, it uses an FX correlator rather than a multiplication interferometer to improve sensitivity by a further factor of $2^{1/2}$ and provides the ability to perform self-cal to vastly improve the dynamic range of maps, has ten times the bandwidth (30 vs 3 MHz), and can simultaneously make maps, coherently dedisperse multiple pulsars in the primary beam, search coherent fan beams for dispersed single pulses out to dispersion measures of several 1000 pc cm$^{-3}$, excise radio frequency intereference and 
