We describe the deployment of a modern cluster of server-class computers at the Molonglo Observatory Synthesis Telescope (MOST) that have transformed the instrument into a powerful new facility (the UTMOST). The servers form part of a hybrid solution to the observatory's radio astronomy signal processing problem. The emphasis on software and commodity-off-the-shelf hardware has enabled extremely rapid deployment and re-use of many powerful existing "software backends" already proven at other observatories. 

The UTMOST provides a dramatic increase in the scientific capabilities of the MOST. Its primary beam has four times the field of view and it uses an FX correlator rather than a multiplication interferometer to improve sensitivity by a further factor of $2^{1/2}$. The ability to perform self-cal vastly improves the dynamic range of maps and it can produce an arbitrary number of spectral channels.\\\\ Inherent to the solution is the ability to run commensal modes meaning that it can simultaneously make maps, coherently dedisperse multiple pulsars in the primary beam, search coherent fan beams for dispersed single pulses out to dispersion measures of several 1000 pc cm$^{-3}$, excise radio frequency interference and offer a range of diagnostic information previously unobtainable. The cluster can record and playback raw voltages to facilitate rapid debugging and the trials of new modes. It can also sub-divide the telescope into up to 352 independent modules that can be used to measure the beam-shape and performance of individual sections that help improve maintenance efficiency. A pulsar timing program has already commenced, and the telescope will soon begin searching for fast radio bursts, rotating radio transients (RRATs), and performing synthesis imaging. 
