\subsection{The discovery of the Fast Radio Bursts}

Well after the SKAMP-2 project commenced, a new class of radio transient emerged termed "Fast Radio Bursts" or FRBs. \cite{Lorimer_2007} reported the first FRB, a dispersed 30-Jy pulse of just 5 milliseconds in duration. This FRB was found in archival data from the Parkes 64\,m radio telescope taken some 7 years earlier. Intensive follow-up of the FRB phenomenon resulted in \citep{Thornton_2013} reporting four more FRBs, with dispersion measures ranging up to 1100 pc cm$^{-3}$, with these authors proposing they are potentially of extragalactic origin. \citep{Thornton_2013} estimated a rate of FRBs of 10$^4$/sky/day. If extragalactic, they offer great potential for probing the ionised Intergalactic Medium (IGM), apart from being of major interest in themselves. But finding FRBs is extremely difficult \cite{Keane_2012} -- only 8 are now in the literature, and all have been detected by single dishes. As a consequence, their spatial localisation on the sky is typically limited to no better than tens of arcminutes, insufficient to unambiguously link them to a putative host galaxy. 

FRBs are not without their controversial aspects. \cite{Burke_Spolaor_2011} detected similar swept signals most likely of a terrestrial origin at the Parkes telescope (dubbed "Perytons") to FRBs, raising doubts about the Lorimer burst's celestial origin. These signals are now confirmed as terrestrial \cite{2015arXiv150402165P}, and are seen in all beams of the Parkes instrument, whereas the \cite{Thornton_2013} sources only appear in one beam of the Parkes telescope, and had a range of dispersion measures rather different to that of perytons.

Intriugingly, in the late 1980s, a transient event detector on the MOST recorded a series of impulsive millisecond-timescale radio bursts. (Amy 1989) tentatively concluded that they originated many 1000 km away, and at a rate of order 18,000 times per sky per day. MOST’s low frequency resolution (3 MHz) meant that their dispersion properties could not be measured but the event rate is similar to that measured by (Thornton et al., 2013) at 1.4 GHz (104/sky/day).

An important way to unambiguously confirm the celestial nature of the FRBs would be to detect them with an interferometer like MOST, so that distance limits can be placed on their point of origin.
    