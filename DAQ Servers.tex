\subsection {DAQ Servers}

The data acquisition (DAQ) servers are commodity off the shelf server-class machines with a 6-core Intel CPU, 64 GB of RAM, a 128 GB SSD drive, a dual-port 10Gb ethernet Network Interface Card (NIC), a 56 Gb Mellanox Infiniband card and a GTX 690 graphics card (GPU). The {\bf psrdada} software library is used to catch the UDP packets. Upon start-up, typically a few packets are dropped but once the CPU core catching the data is busy, the system will usually not drop any after the first second of a given observation. 

Within a few milliseconds of starting an observation the DAQ machines use the {\bf psrdada} software library to catch 8K packets from the PFB with almost no dropped packets. The data are transferred from the NIC to RAM, where an finite impulse response filter in the GPU is applied to compensate for delays (both system and geometric), and an analysis of the spectral kurtosis of the incoming signal is applied to help detect deviations from ``normal'' statistics. If the voltages are suspect, they are replaced by gaussian random noise of the same power as the input signal. This eliminates the worst of the phone calls incident upon the telescope. For a limited ($\sim50$\%) fraction of the modules, it is possible to detect and fold the data at the topocentric period of a radio pulsar using the {\sc dspsr} software.