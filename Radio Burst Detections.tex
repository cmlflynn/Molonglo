\subsection{Fast Radio Burst and Single Pulse Searching (Manisha?)}

Currently there is insufficient computer power to form the fan beams, dedisperse them all and search for Fast Radio Bursts. We are currently benchmarking the latest GPUs, and it should soon be possible to deploy the 8 remaining beam-forming servers with multiple GPUs in each that can:
\begin{itemize}
\item Form the fan beams.
\item square law detect them and integrate to $\sim$ ms timescales.
\item dedisperse the fan beams and search for FRBs and single pulses from pulsars in real time out to a DM of $\sim$2000 pc cm$^{-1}$.
\end{itemize}

The sensitivity of the array is such that it should be capable of discovery O(100) FRBs every year. Although the sensitivity at the beam centre will only be $\sim$35\% of the Parkes telescope, it will have almost 20 times the field of view and be used to search for bursts almost 100\% of the time. The FRBs it finds will have a positional error only 43$''$/$SNR$ (where $SNR$ is the signal to noise ratio) in the east-west direction but about 2 deg in the north-south direction.