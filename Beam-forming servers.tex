\subsection{Beam-forming}

After the corner-turn, the data are aligned in time in the RAM of the beam-forming servers, and all of the same frequency channels are co-located. At this point the servers can perform one of five different operations. They can detect and sum each voltage to form a single \textit{incoherent} primary beam, add the complex voltages to form a single \textit{coherent} tied array beam, cross correlate each input against the others an ``X'' mode, or produce fine channels that are themselves correlated (an ``FX'' mode), or, in the future, use the fast fourier transform (FFT) to create fan beams. 