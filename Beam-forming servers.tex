\subsection{Beam-former (BF) servers}

After the corner-turn, the data are aligned in time in the RAM of the beam-forming servers, and the same frequency channels are co-located. At this point the servers can perform one of five different operations: detect and sum each voltage to form a single \textit{incoherent} primary beam; add the complex voltages to form a single \textit{coherent} tied array beam; cross correlate each input against the others (an ``X'' mode) to form fringes; produce fine channels that are themselves correlated (an ``FX'' spectral-line mode); and/or  create multiple fan beams on the sky. 

The power from the incoherent primary beam can be written to a filterbank file or folded into a pulsar profile. The voltages from a tied array beam can be detected and written to a filterbank file for offline processing, or incoherently or coherently dedispersed and folded into a pulsar profile. A real-time fan-beam creation and dispersed pulse searching mode is under development for deployment on the new beam-forming machines and is currently limited to one beam at a time.

Map making has been performed already from fringes using standard packages such as miriad(REFERENCE). maps can also be made from the timeseries data produced in filterbank format from the multiple fanbeams. 
  