\subsection{The MOST as an FRB detector}
The SKAMP-2's wide field of view (about 4.25$\times$2.1 square degrees), large collecting area (18,000 m$^2$) and compact configuration made it an attractive instrument with which to search for FRBs. Most interferometers cannot hope to form the number of coherent beams required to tile the entire primary beam, dedisperse them and search for FRBs. The MOST is almost perfect for this task however as its filling factor is close to unity. 

The radiometer equation provides an estimate of the system noise $N$ in an integration of length $t$ 
\begin{equation}
N={{T_{\rm sys}}\over{G\sqrt{BN_{\rm p}t}}}
\end{equation}
\noindent where $B$ is the bandwidth, $N_{\rm p}$ is the number of polarisations (for the MOST $N_{\rm p}=1$)and $G$ is the gain of the antenna in K Jy$^{-1}$. For the MOST $G\sim 3.5$ K Jy$^{-1}$. The effective bandwidth is considerably less than the 31.25 MHz that is processed due to optimisation for the previous 3 MHz band. In an observation of duration $t=5$\, ms if we conservatively assume $T_{\rm sys}$=60K and $B=15 $ MHz the noise is $\sim$60 mJy. This is enough to detect the brightest FRB in Thornton et al. (2013) at about the 8 sigma level but the UTMOST has about 230 times the field of view of a single beam of the 13-beam Parkes receiver at 20cm. The Parkes telescope has about 17 mJy of noise in a 5 ms integration. In an isotropic universe, we might except the event rates to scale as the sensitivity $S^{-3/2}$ and the field of view. Since Parkes is finding one event every 10 days of on-sky time, the event rate for Molonglo may be one every 4 days or so. The Parkes telescope is not a dedicated FRB search instrument however, and only finds a few FRBs per year. If Molonglo can search for 24 hours a day, it should be a prolific contributor to FRB discovery.
