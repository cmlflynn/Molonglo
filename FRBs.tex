\subsection{A fillip for SKAMP: The MOST as an FRB detector}

The SKAMP-2's wide field of view (about 4.25$\times$2.1 square degrees), large collecting area (18,000 m$^2$) and the telescope's high filling factor (near unity) are an unusual property set for a radio telescope, and are as if tailor-made for FRB searches. Most interferometers can come nowhere near being able to tile the entire primary beam with coherent beams, dedisperse the time-series data from each, and search for FRBs, because the processing requirements are enormous. MOST excels at this task.  

The radiometer equation gives the system noise $N$ in an integration of length $t$ 
\begin{equation}
N={{T_{\rm sys}}\over{G\sqrt{BN_{\rm p}t}}}
\end{equation}
\noindent where $B$ is the bandwidth in Hz, $T_{\rm sys}$ is the system temperature in K, $N_{\rm p}$ is the number of polarisations (for the MOST $N_{\rm p}=1$) and $G$ is the antenna gain in K Jy$^{-1}$. 

For the MOST, the antennae gain is $G\sim 3.5$ K Jy$^{-1}$, for all 352 modules added coherently. The effective bandwidth is considerably less than the 31.25 MHz that is processed and sent to our UTMOST backend, primarily due to the ring antennae and resonant cavity responses of the 3 MHz system, and is presently closer to 15 MHz (June 2015). 

For a burst of width $t=5$\, ms if we conservatively assume $T_{\rm sys} = 60$ K and $B = 15$ MHz, the noise is figure is $\sim 60$ mJy. This is low enough to permit detection of the brightest FRB published in Thornton et al. (2013) at about the 8 sigma level at UTMOST. UTMOST's advantage is that it has $\sim 230$ times the field of view of a single beam of the 13-beam Parkes receiver at 20 cm. The noise figure at Parkes is 17 mJy for a 5 ms integration, and we will make the standard assumption for flux limited searches of standard candles that the event rate scales as the survey sensitivity $S^{-3/2}$, and of course the field of view. Under these assumptions, since Parkes finds one event approximately every 10 days of on-sky time, the event rate for Molonglo operating at maximum sensitivity is of order one every 4 days -- assuming a flat spectral index between 1.4 GHz (Parkes) and 843 MHz (Molonglo). More detailed computations of the event rate at Molonglo, including models for the co-moving density of FRBs in a standard $\Lambda$CDM universe, have been made by \cite{Caleb2015}, who show that the survey sensitivity scales closer to $S^{-0.9}$ when cosmological, dispersion and event scattering factors are accounted for, favouring detection rates at Molonglo. In any case, since the Parkes telescope is in no sense a dedicated FRB search instrument, only a few FRBs are found per year (typically commensally while pulsar search programs are operating), whereas Molonglo can search for FRBs essentially 24 hours a day, so it has the potential to become a dominant contributor to FRB discovery, measuring if they are in the near-filed or distant, and their spectral properties. 

Retention of the raw voltages used that to discover FRBs would enable us to measure East-West positional accuracy of $\sim 43''/S$, where $S$ is the signal-to-noise ratio of the event. Unfortunately the error box extends $\sim$2.1$^{\circ}$ in the North-South direction, so that the identification of FRBs with extragalactic sources would be almost as problematic as with single dish telescopes. FRBs seen at UTMOST would be very exciting, as the interferometer offers the chance to measure a source parallax, confirm their presence at lower frequencies, and substantially increase their discovery rate. 
  
  