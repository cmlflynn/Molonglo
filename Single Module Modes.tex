\subsection{Single Module Modes}
\subsubsection{Incoherent Pulsar Timing Mode}

Observations of bright pulsars were made on each module of the telescope individually, prior to our being able to phase the telescope, which permitted timing of pulsars with the module data added incoherently. 

In Figure XX we show pulsar timing residuals from the millisecond pulsar PSR J0437-4715, in incoherent mode, over a period of 20 days. The RMS residuals are already just 2$\mu$s at a time when the array was only operating at a few percent of theoretical maximum performance. The Parkes Pulsar Timing Array ephemeris was used and it was not necessary to fit for any parameters to achieve these RMS residuals.

With the advent of our capability to form a tied-array beam using an arbitrary number of modules, incoherent timing mode has been effectively superseded. We nevertheless retain the capability to observe in individual module mode, as it is quick means of characterising the response of each module independently of phased array modes. This has proved very useful during the commissioning phases, for example as a means of identifying poorly performing modules and how effectively RFI amelioration strategies work on a module by module basis.     



  
  